\documentclass[a4paper, 10pt]{article}
\usepackage[german,english]{babel}
\usepackage{graphicx} 
\usepackage{verbatim}
\usepackage{booktabs}
%\usepackage{listings} 
\usepackage[latin1]{inputenc} 
%\usepackage{psfrag} 
%\usepackage{color}
\begin{document} 
\fontsize{8pt}{8pt} 
\bfseries
\sf
%---------------------------------------------------------------------------------------------------------------
{\huge {\bf HOOK Analysis}}\\ \\ \\
%\begin{verbatim} intensity-file \end{verbatim}\par
%\end{flushright}
% Namen der Personen
%---------------------------------------------------------------------------------------------------------------
\section{Chip}
\begin{tabular}{lr}
	ID of the chip &\verb| chipId | \\
	Date of analysis &\verb| startingTime| \\
\end{tabular}\\	
%---------------------------------------------------------------------------------------------------------------
%--Chip characteristics----------------------------------------------------------------------------------------- 
%--------------------------------------------------------------------------------------------------------------- 
\section {Chip characteristics} 
\begin{tabular}{lr}
	{\bf Optical background} \\
	Average intensity of the Optical Background &
  backgroundSubtractionBgMean \\ \\
	{\bf Hookcurve} \\
	Breakpoint (hookplot uncorrected) & (Hookplot-Primary-intersectionPointXvalue , Hookplot-Primary-intersectionPointYvalue)\\
	Breakpoint $(\Sigma_{0},\Delta_{0})$ (hookplot corrected) & (Hookplot-Corrected-intersectionPointXvalue , Hookplot-Corrected-intersectionPointYvalue)\\
	Measured height of the hook (uncorrected) &\verb| hookcurveHeightUncorrected| \\
	Measured height of the hook (corrected) &\verb| hookcurveHeightCorrected| \\
	Measured width of the hook (uncorrected) &\verb| hookcurveWidthUncorrected| \\
	Measured width of the hook (corrected) &\verb| hookcurveWidthCorrected| \\ \\

	{\bf Non-specific range} \\
    Fraction of non-specific probe sets (uncorrected) & probesetNsFractionUncorrected \\
    Fraction of non-specific probe sets (corrected) & probesetNsFractionCorrected \\
  	%NS ratio Probeset level & probesetNsFraction \\
    %Average non-specific background for PM & nsMeanProbesPmCorrected \\
	%Average non-specific background for MM & nsMeanProbesMmCorrected \\ 
	Width of non-specific Range (uncorrected) & hookcurveNsRangeWidthUncorrected \\
	Width of non-specific Range (corrected) & hookcurveNsRangeWidthCorrected \\ \\
	%Standard deviation of non-specific background for PM & nsStdProbesPmCorrected \\
	%Standard deviation of non-specific background for MM & nsStdProbesMmCorrected \\ \\
	{\bf Specific range (hook fit)} \\
  Saturation intensity Imax (log(M))&\verb| saturationImax| \\
	PM/MM-gain (log(s))  &\verb| nsMeanProbessetsDiffCorrected(logB)| \\ %PM/MM ratio
	Calculated height of the hook $\alpha$ (only if chip saturates) &\verb| saturationA| \\
	Calculated width of the hook $\beta$ (only if chip saturates) &\verb| saturationF| \\ 
    %Kink point & hookcurveKinkPoint \\
	Mean expression R$|$R$>$0.5 &\verb| rGreaterThan05Mean| \\ 
	Mean expression log(R+1)$|$R$>$0.5
  &\verb| rGreaterThan05LogPlus1Mean| \\ \\

	{\bf Degradation} \\
  Improved 3',5' ratio $\frac{\langle \Sigma^{3'} \rangle_{sp(s)}}{\langle \Sigma^{5'} \rangle_{sp(s)}}$ 
  &\verb| DegradationBiasAverageSpecfific| \\
  Tongs-Measure &\verb| DegradationBiasMario| \\ \\    
\end{tabular}\\
\begin{tabbing}
	{\bf Additional Information}\\
	  Chip-type \= (sequence file) \\  \verb| chip-file| \\
    Number of probes (probe count):  probeCount \\ \\
    %ExperimentCode \\ \\
\end{tabbing}
{\bf Statistics of the NS-range} \\
\begin{tabular}{|l||r|r|r|r|} %\begin{tabular}{l||r|r|r|r|r}

    \hline \hline
    \rule{0pt}{10pt}&$PM_{uncorr}$ & $MM_{uncorr}$ & $PM_{corr}$ & $MM_{corr}$ \\
    \hline
    \rule{0pt}{10pt}$\mu$ Probes & nsMeanProbesPmUncorrected & nsMeanProbesMmUncorrected & nsMeanProbesPmCorrected & nsMeanProbesMmCorrected \\
    \hline
    \rule{0pt}{10pt}$\mu$ Probesets & nsMeanProbessetsPmUncorrected & nsMeanProbessetsMmUncorrected 
                    &  nsMeanProbessetsPmCorrected & nsMeanProbessetsMmCorrected \\
    \hline
    
    \rule{0pt}{10pt}$\sigma$ Probes & nsStdProbesPmUncorrected & nsStdProbesMmUncorrected & nsStdProbesPmCorrected & nsStdProbesMmCorrected \\
    \hline
    \rule{0pt}{10pt}$\sigma$ Probesets & nsStdProbessetsPmUncorrected & nsStdProbessetsMmUncorrected 
                       & nsStdProbessetsPmCorrected & nsStdProbessetsMmCorrected \\
    \hline
    \rule{0pt}{10pt}$\rho$ NS probes & \multicolumn{2}{|c|}{ nsCorrelationProbesUncorrected } & \multicolumn{2}{|c|}{nsCorrelationProbesCorrected} \\
    \hline
    \rule{0pt}{10pt}$\rho$ NS probesets & \multicolumn{2}{|c|}{ nsCorrelationProbesetsUncorrected } & \multicolumn{2}{|c|}{nsCorrelationProbesetsCorrected} \\
    \hline
\end{tabular}\\
	$\mu$ mean
	$\sigma$ standard deviation
	$\rho$ correlation coefficient 
%--------------------------------------------------------------------------------------------------------------- 
%--Plots (Hook Curves and Sensitivity profiles)----------------------------------------------------------------- 
%--------------------------------------------------------------------------------------------------------------- 
\newpage
\section {Plots}  

\subsection*{Hook Curves}
	The Hook Curves gives a general and intuitive overview of "how your expreiment performed".Below you will find two versions of the 
	so-called Hook Curves, first the Hook Curve created from uncorrected intensities and second that from corrected data.\\
	\includegraphics[scale=0.4]{Hookplot-Primary.png} \\
	Intersection point: ( Hookplot-Primary-intersectionPointXvalue , Hookplot-Primary-intersectionPointYvalue ) (hookplot primary)\\ \\
	\includegraphics[scale=0.4]{Hookplot-Corrected.png} \\ 
	Intersection point: ( Hookplot-Corrected-intersectionPointXvalue , Hookplot-Corrected-intersectionPointYvalue ) (hookplot corrected)\\

\newpage
\subsection*{Sensitivity Profiles}
	The Sensitivity Profiles reflect the relative binding strength of each base at each position in the probe sequence.
	This behaviour is different for PM and MM sequences as well as for specific and non-specific hybridization.\\
	{\bf Note:} Here we allways show the single-base related Sensitivity Profiles.\\ \\ \\
	Non-specific hybridization for PM and MM probes \\
  %  Variant ending with 0 for additive profile types
  % \IfFileExists{SensitivityProfileNsPm-Corrected-single0.png}{\includegraphics[scale=0.25]{SensitivityProfileNsPm-Corrected-single0.png}}{} 
  % \IfFileExists{SensitivityProfileNsMm-Corrected-single0.png}{\includegraphics[scale=0.25]{SensitivityProfileNsMm-Corrected-single0.png}}{}
	\IfFileExists{SensitivityProfileNsPm-Corrected-single.png}{\includegraphics[scale=0.25]{SensitivityProfileNsPm-Corrected-single.png}}{} 
	\IfFileExists{SensitivityProfileNsMm-Corrected-single.png}{\includegraphics[scale=0.25]{SensitivityProfileNsMm-Corrected-single.png}}{} \\ \\ \\
	%\includegraphics[scale=0.4]{SensitivityProfileNsMm-Corrected-single.png} \\ \\
%\newpage	
	Specific hybridization for PM and MM probes \\
	\IfFileExists{SensitivityProfileSPm-single.png}{\includegraphics[scale=0.25]{SensitivityProfileSPm-single.png}}{} 
	\IfFileExists{SensitivityProfileSMm-single.png}{\includegraphics[scale=0.25]{SensitivityProfileSMm-single.png}}{}\\
	%\includegraphics[scale=0.25]{SensitivityProfileSMm-single.png} \\

\IfFileExists{PolymeraseBias.png}{
  \subsection*{Degradation Plot}
  The tongs-plot displays the degradation effect subject to probeset
  average intensities. \\
  \includegraphics[scale=0.35]{PolymeraseBias.png} \\
}{}

%---------------------------------------------------------------------------------------------------------------
%--Parameter Settings------------------------------------------------------------------------------------------- 
%--------------------------------------------------------------------------------------------------------------- 
\newpage
\section{Parameter settings}
\begin{tabbing}
\input{parameters}
\end{tabbing}
\end{document}
